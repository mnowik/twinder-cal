\chapter*{Introduction}
\addcontentsline{toc}{chapter}{Introduction}

\paragraph{}
Since the sudden upsurge of Social Networks at the beginning of the 2000's, the time spent on social media
has growing considerably. In 2006, two years after Facebook's creation, users were connected for only 5.5
min. per day. Five years later, the number of users has been multiplied by 70 and the time spent by 2.5. With
these increases social networks have become very appealing to several sectors, such as marketing,
insurance and investing. However, social medias have to deal with many constraints coming from from their
users and the internet itself. Today, there exists more than twenty virtual communities with more than 100
million active users. As a result, there is a vast amount of information available to the public, and this is
increasing at an ever-increasing rate. In 1970, Alvin Toffler talked about Information overload in cognitive
psychology. Over the decades, following the technology evolution, this expression was more and more
applicable to the web and to social media. In addition, companies like Twitter or Facebook have to deal with
the engagement of their users. Despite its 500 million active users, the microblogging service saw its stock
price drop of more than 23\% because of low engagement of its users. The time spent online is now
considered more important than the number of users.\\
After having grown their number of users, the social media companies are now focusing on increasing the
engagement and the time spent on their sites. In order to retain users, Facebook conducted research in the
field of sociology linked to big data analysis. Thanks to this research, the EdgeRank algorithm was launched
in 2010, and was designed to optimize users News Feed. In 2012, Facebook launched its new design in
order to improve the user experience and the workflow of its platform. Following its big brother, Twitter
launched in its new design in April 2004. The design was considered by many bloggers as the copy of
Facebook's one. However, Wall Street answered positively to this strategy which reassured many investors.
In this project, we perform an A/B test on Twitter users in order to measure their engagement. In order to
avoid Information overload, we use a simple design which allows the candidate to engage with their tweets
one by one. Our two dependent variables are interest and time, we attempt to measure their impact on the
engagement in a non overloaded environnement.\\
Firstly, the paper discusses the studies related to this field and explain the motivation this topic is outlined.
Second, the paper describes the protocol which was followed and the technical implementation done in
Python. Third, the method of data acquisition is recalled. Finally, we analysis and discuss the data is
analysed and discussed in order to come to a conclusion.

