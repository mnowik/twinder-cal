\chapter*{Introduction}
\addcontentsline{toc}{chapter}{Introduction}

\paragraph{}
Since the sudden upsurge of Social Networks at the beginning of the 2000's, the time spent on social media has grown considerably. Today, they have a worldwide impact and  influence many fields, both social and economic. However, social networks have to overcome the restrictions that come from their users, the web or social medias themselves. For example, Facebook admitted last year that they had some difficulty with engaging youth (ages 10-16) \cite{f_youth_engagement}, which is directly linked to the large number of virtual communities. There are more than twenty social media platforms/social networking sites with a total of over 100 million users. As a result, there is a vast amount of information available to the public public that does not remain in the hands of one single network. These consequences have a direct impact on the confidence that investors have in companies like Twitter or Facebook. Despite its 500 million users, the microblogging service saw its stock price drop more than 23\% \cite{t_stocks} in February due to low user engagement. This new metric is not as easy to measure as the number of users and can have different aspects depending on the media \cite{s_engag_measure}. \\
Social media companies are now focusing on increasing user engagement and the time spent on their sites. In order to retain users, Facebook conducted research in the field of sociology linked to big data analysis. This research resulted in the EdgeRank algorithm being launched in 2010, whose main goal was to avoid overwhelming users with the News Feed. More recently, Twitter launched in its new design in April 2014. The design had good reception because it puts the user back in the center of the application. This strategy reassured many investors who saw this move as a desire to increase website engagement.\\
In this project, we simplified the engagement of the user by his or her engagement time. We created an application  that shows tweets one by one and measures the time spent on each tweet, as well as the interest or not for it. In order to compare the two factors, time and interest, we performed two A/B test on different users. Our two independent variables are the number of tweets liked and the time spent on tweets during the test A. The first one is recorded by a binary classification of the tweets shown one by one. The second one is the time that the user take to make the previous decision. This design allows use to measure to dependent variables that are the interest and the time engagement. \\
Firstly, the paper discusses the studies related to this field and explains the relevance of this topic. Second, the paper describes the protocol that was followed and the technical implementation done in Python. Third, the method of data acquisition is described. Finally, the data is analysed and discussed, allowing us to arrive at our conclusion.

