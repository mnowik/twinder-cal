\chapter*{Conclusion}
\addcontentsline{toc}{chapter}{Conclusion}

\paragraph{}
This multidisciplinary thesis gave us a better understanding of user behavior on Twitter. The application incorporated Human Computer Interaction, User Experience and sociological methods. We first wanted to create an application which increased the time engaged on series of Tweets seen one-by-one. The two A/B tests made it possible to check the number of tweets 'liked' or the time spent on tweets during the first series and if they had an important influence on the interest and the time engaged on the second series. With Twinder, the web application developed for the project, it was possible to display tweets one-by-one while recording the time spent on them and the decision made to 'like' or 'dislike' each tweet. While many others studies have been done on this subject, their focus has centered on levels of engagement with the content of the tweets. With our design, it was possible to precisely measure the time of engagement for each tweet.
After running the experiment on 15 users, we analyzed the results of the application and the survey. Despite the few numbers of candidates, we could see clear differences between the test-like-B and test-time-B. While it is easier to increase the interest of the users, as demonstrated by test-like-B, this interest strongly affects engagement. The users spend much less time on the series and feel more tired at the end of this test. The opposite is seen, however, when people spend almost the same time on test-time-A and test-time-B despite the decrease in interest for the second series. \\ 
This project also demonstrated the different ways people are using Twitter. We identified two categories: the superficial user and the careful user. These behavioral differences are essential because it shows the limits of our experience. \\
The levels user engagement seem to be the new metric for emerging, massive social networks. However, it is an extremely difficult constant to measure. Depending on social networks, we can link the engagement to the engaged time, the number of 'likes', and numbers of retweets. The irregularity of the engagement makes it difficult to produce strong results. However, as we mentioned in our discussion, it would be interesting to measure this user focus with an EEG. These brainwaves could directly give the users' state after each tweet, and they would act as a better metric for engagement. A redesign of the methods based on the brainwaves and the solutions given in the discussion would be a good starting point to increase user engagement on Twitter. 