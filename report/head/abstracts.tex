%\begingroup
%\let\cleardoublepage\clearpage



% English abstract
\cleardoublepage
\chapter*{Abstract}
%\markboth{Abstract}{Abstract}
\addcontentsline{toc}{chapter}{Abstract} % adds an entry to the table of contents

Once social medias have important number of users like Facebook, Twitter or Linkedin, the new metric in order to measure their progress is the engagement. This value is not easy to measure and does not have an unique formula, each media has is own based on the feature that it has: like, retweet, favorite, etc. Moreover, it is really hard for companies to measure the time spent on unit post like tweets.
With this work, we would like to understand better the engagement on Twitter and then try to increase the interest and the time spend on the platform.
To solve this problem, we created a new way to visualize tweets one-by-one. Due to Twinder, the web application dedicated, it was possible to create an A/B test which record the time spent on tweets and the decision of users of they like or not each tweets.
One unique calibration test A (series of 50 tweets) allows us to generate two tests B which half of the candidates are doing. In test-like-B, more a source of tweets was liked in test A more we are going to displays tweets from it. In test-time-B, more users spent time on tweets coming from a source, more we are going to display tweets from it. Then, we use the time spend on tweets and the number of likes of the test A to generate tests B.
After running this experiment and creating a survey, we realized that people in both configurations test-A/test-like-B and test-A/test-time-B are spending less time in the second series than the first one. However, in the first configuration we noticed a decrease of 24.20\% of the time spent, compare to a decrease of only 2.51\% in the second one. At the same we measured the number of likes per series which is a good indicator of interest. At the opposite, users who did test-like-B were way more interest in this series that users who did test-time-B. Indeed, the first configuration received 18.52\% more likes than test A and the second configuration received 9.14\% less likes than test A. In parallel, the results of the survey allowed us to see that users who did test-like-B were more tired than the other even if they seems more interested.
We did not succeed in increasing the engagement on a series of Tweets, but we notice that it is easy to increase the interest of a series. However, the interest cost a lot for the user who is more tired and spend less time in test-like-B. Finally, we found a way to keep the time engaged almost stable (-2.51\%) despite the decrease of the interest (-9.14\%).

\vskip0.5cm
Key words: Engagement, Interest, HCI, User Experience.
